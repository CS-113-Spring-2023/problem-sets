\documentclass[10pt]{article}
 
\usepackage[margin=1in]{geometry} 
\usepackage{amsmath,amsthm,amssymb, graphicx, multicol, array}

 
\newcommand{\N}{\mathbb{N}}
\newcommand{\Z}{\mathbb{Z}}
 
\newenvironment{problem}[2][Problem]{\begin{trivlist}
\item[\hskip \labelsep {\bfseries #1}\hskip \labelsep {\bfseries #2.}]}{\end{trivlist}}

\newenvironment{definition}[2][Definition]{\begin{trivlist}
    \item[\hskip \labelsep {\bfseries #1}\hskip \labelsep {\bfseries #2.}]}{\end{trivlist}}

\begin{document}

\title{CS/MATH 113 -- Problem Set 1}
\author{Habib University -- Spring 2023}
\date{Week 01}
\maketitle

\section{Definitions}


\begin{definition}{1}(Integer)
    An integer is a number with no decimal or fractional part and it includes negative and positive numbers, including zero.
\end{definition}


\begin{definition}{2}(Even Integer)
    An integer is even if it can be written as $2k$ where $k$ is an integer.
\end{definition}

\begin{definition}{3}(Odd Integer)
    An integer is even if it can be written as $2k+1$ where $k$ is an integer.
\end{definition}

\begin{definition}{4}(Parity)
    The parity of an integer is its property of being even or odd.
\end{definition}

\begin{definition}{5}(Natural Numbers)
    Natural numbers are a set of positive numbers from $1$ to $\infty$
\end{definition}

\begin{definition}{6}(Rational Numbers)
    Rational numbers are any numbers that can be expressed in the form $\frac{a}{b}$ where $a$ and $b$ are integers, and $b \neq  0$
\end{definition}

\begin{definition}{7}(Divisiblity)
    A nonzero integer $m$ divides an integer $n$ provided that there is an integer $q$ such that $n=mq$. We say that $m$ is a divisor of $n$
    and that $m$ is a factor of $n$ and use the notation $ m | n$
\end{definition}
\section{Problems}
Using the definitions above solve the following problems.
\begin{problem}{1}
Prove that the sum of two odd integers is even.
\end{problem}
\begin{problem}{2}
Prove that the product of two even integers is even.
\end{problem}
\begin{problem}{3}
Prove that the product of any two rational numbers is also a rational number.
\end{problem}
\begin{problem}{4}
Prove that the square of any natural number is also a natural number.
\end{problem}
\begin{problem}{5}
Prove that the square of any rational number is also a rational number.
\end{problem}
\begin{problem}{6}
In each case either prove the statement or find a counterexample.
\begin{itemize}
    \item[(a)] The sum of any three consecutive integers (positive or negative) is divisible by $3$.
    \item[(b)] The product any two even integers is divisible by $4$.
    \item[(c)] The product of any four consecutive integers (positive or negative) is divisible by $8$.
    \item[(d)] If $a - b$ has remainder 0 when divided by $m$, then $a$ and $b$ have remainders $0$ when divided by $m$.
    \item[(e)] If $n$ is an odd integer, then $3n + 3$ is divisible by $6$
\end{itemize}
\end{problem}
\begin{problem}{7}
Prove that the product of five consecutive integers is divisible by 120.
\end{problem}
\begin{problem}{8}
Prove that the sum of two postive integers of the same parity (odd/even) is even.
\end{problem}
\begin{problem}{9}
Prove or disprove that if $a+b$ is an odd integer, then both $a+x$ and $b+x$ are odd integers,
where $a,b$, and $x$ are integers.
\end{problem}




\end{document}
