\documentclass{article}

\usepackage{url}
 
\usepackage{geometry} 
\usepackage[shortlabels]{enumitem}
\usepackage{amsmath,amsthm,amssymb, graphicx, multicol, array}

 
\newcommand{\N}{\mathbb{N}}
\newcommand{\Z}{\mathbb{Z}}
\newcommand{\olsi}[1]{\,\overline{\!{#1}}} % overline short italic

 
\newenvironment{problem}[2][Problem]{\begin{trivlist}
\item[\hskip \labelsep {\bfseries #1}\hskip \labelsep {\bfseries #2.}]}{\end{trivlist}}

\newenvironment{definition}[2][Definition]{\begin{trivlist}
    \item[\hskip \labelsep {\bfseries #1}\hskip \labelsep {\bfseries #2.}]}{\end{trivlist}}

\begin{document}

\title{Problem Set 5\\CS/MATH 113 Discrete Mathematics}
\author{Habib University | Spring 2023}
\date{Week 06}
\maketitle

\section{Problems}

\begin{problem}{1}[Chapter 2.1, Question 12]
Determine whether these statements are true or false.
\begin{enumerate}[(a)]
    \item $ \phi \in \{\phi\}$
    \item $ \phi \in \{ \phi,\{\phi\} \}$
    \item $ \{ \phi \} \in \{ \phi \} $
    \item $ \{ \phi \} \in \{\{ \phi \}\} $
    \item $ \{ \phi \} \subset \{ \phi, \{\phi \} \} $
    \item $ \{ \{ \phi \} \} \subset \{ \phi, \{ \phi \}\} $
    \item $ \{ \{ \phi \} \} \subset \{ \{\phi \}, \{ \phi \}\} $
\end{enumerate}
\end{problem}

\begin{problem}{2}[Chapter 2.1, Question 23]
Find the power set of these sets where $a$ and $b$ are distinct elements.
\begin{enumerate}[(a)]
    \item $\{a\}$
    \item $\{a,b\}$
    \item $\{ \phi, \{ \phi \}\}$
\end{enumerate}
\end{problem}

\begin{problem}{3}[Chapter 2.1, Question 24]
Can you conclude that $A = B$ if $A$ and $B$ are two sets with the same power set ?
\end{problem}
\begin{problem}{4}[Chapter 2.1, Question 25]
How many elements does each of these sets have where $a$ and $b$ are distinct elements ?
\begin{enumerate}[(a)]
    \item $ \mathcal{P}(\{a,b,\{a,b\}\})$
    \item $ \mathcal{P}(\{\phi,a,\{a\}, \{\{a\} \}\})$
    \item $ \mathcal{P}(\mathcal{P}(\phi))$
\end{enumerate}
\end{problem}
\begin{problem}{5}[Chapter 2.1, Question 27]
Prove that $ \mathcal{P}(A) \subseteq \mathcal {P}(B)$ if and only if $ A \subseteq B$
\end{problem}
\begin{problem}{6}[Chapter 2.1, Question 28]
Show that if $A \subseteq C$ and $ B \subseteq D$, then $ A \times B \subseteq C \times D$
\end{problem}
\begin{problem}{7}[Chapter 2.1, Question 29]
Let $A = \{a,b,c,d\}$ and $B = \{x,y\}$. Find
\begin{enumerate}[(a)]
    \item $ A \times B$
    \item $ B \times A$
\end{enumerate}
\end{problem}
\begin{problem}{8}[Chapter 2.1, Question 40]
Show that $ A \times B \neq B \times A$, when $A$ and $B$ are nonempty, unless $ A = B$
\end{problem}
\begin{problem}{9}[Chapter 2.1, Question 44]
Prove or disprove that if $A,B$, and $C$ are nonempty sets, and $A \times B = B \times C$, then $B=C$
\end{problem}
\begin{problem}{10}[Chapter 2.2, Question 5]
Prove the complementation law in Table 1 by showing that $ \bar{\bar{A}} = A$
\end{problem}
\begin{problem}{11}[Chapter 2.2, Question 11]
Let $A$ and $B$ sets. Prove the commutative laws from Table 1 by showing that
\begin{enumerate}[(a)]
    \item $A \cup B = B \cup A$
    \item $A \cap B = B \cap A$
\end{enumerate}
\end{problem}
\begin{problem}{12}[Chapter 2.2, Question 19]
Show that if $A,B,$ and $C$ are sets, then $ \olsi{A \cap B \cap C} = \bar{A} \cup \bar{B} \cup \bar{C}$
\begin{enumerate}[(a)]
    \item by showing each side is a subset of the other side
    \item using a membership table.
\end{enumerate}
\end{problem}
\begin{problem}{12}[Chapter 2.2, Question 36]
Prove or disprove that for all sets $A,B$,and $C$ we have
\begin{enumerate}[(a)]
    \item $A \times (B \cup C) = (A \times B) \cup (A \times C)$
    \item $A \times (B \cap C) = (A \times B) \cap (A \times C)$
\end{enumerate}
\end{problem}

\begin{problem}{13}[Chapter 2.2, Question 50]
 Show that if $A$ and $B$ are finite sets, then $ A \cup B$ is a finite set.
\end{problem}
\begin{problem}{14}[Chapter 2.2, Question 51]
    Show that if $A$ is an infinite set, then whenever $B$ is a set, $ A \cup B$ is also an  infinite set.
\end{problem}

\end{document}