\documentclass{exam}

\usepackage{geometry} 
\usepackage{amsmath,amsthm,amssymb, graphicx, multicol, array}


\newcommand{\N}{\mathbb{N}}
\newcommand{\Z}{\mathbb{Z}}

\newenvironment{problem}[2][Problem]{\begin{trivlist}
  \item[\hskip \labelsep {\bfseries #1}\hskip \labelsep {\bfseries #2.}]}{\end{trivlist}}

\newenvironment{definition}[2][Definition]{\begin{trivlist}
  \item[\hskip \labelsep {\bfseries #1}\hskip \labelsep {\bfseries #2.}]}{\end{trivlist}}

\begin{document}

\title{Problem Set 1\\ CS/MATH 113 Discrete Mathematics}
\author{Habib University | Spring 2023}
\date{Week 01}
\maketitle

\section*{Definitions}
\begin{description}
\item [Integer]
  An \textit{integer} is a number with no decimal or fractional part and it includes negative and positive numbers, including zero.
\item[Even Integer]
  An integer is \textit{even} if it can be written as $2k$ where $k$ is an integer.
\item[Odd Integer]
  An integer is \textit{odd} if it can be written as $2k+1$ where $k$ is an integer.
\item[Parity]
  The \textit{parity} of an integer is its property of being even or odd.
\item[Natural Number]
  A \textit{natural number} is any positive numbers from $1$ to $\infty$.
\item[Rational Number]
  A \textit{rational numbers} is any number that can be expressed in the form $\frac{a}{b}$ where $a$ and $b$ are integers, and $b \neq  0$.
\item[Divisibility]
  A nonzero integer $m$ \textit{divides} an integer $n$ provided that there is an integer $q$ such that $n=mq$. We say that $m$ is a \textit{divisor} or \textit{factor} of $n$, denoted as $ m \mid n$.
\end{description}

\hrule

\begin{questions}

  \question Using the definitions above, prove or provide a counterexample for each of the following claims.
  \begin{parts}
    \part The sum of two odd integers is even.
    \part The product of two even integers is even.
    \part The sum of two positive integers of the same parity is even.
    \part Given integers, $a, b, x$, if $a+b$ is odd, then both $a+x$ and $b+x$ are odd.
    \part The square of a natural number is a natural number.
    \part The product of two rational numbers is a rational number.
    \part The square of a rational number is a rational number.
    \part The sum of three consecutive integers (positive or negative) is divisible by $3$.
    \part The product two even integers is divisible by $4$.
    \part The product of four consecutive integers (positive or negative) is divisible by $8$.
    \part The product of five consecutive integers is divisible by $120$.
    \part If $m \mid (a-b) $, then  $m\mid a$ and $m\mid b$.
    \part If $n$ is an odd integer, then $6 \mid (3n + 3)$.
  \end{parts}

\end{questions}

\end{document}
