\documentclass{article}

\usepackage{url}
 
\usepackage{geometry} 
\usepackage{amsmath,amsthm,amssymb, graphicx, multicol, array}

 
\newcommand{\N}{\mathbb{N}}
\newcommand{\Z}{\mathbb{Z}}
 
\newenvironment{problem}[2][Problem]{\begin{trivlist}
\item[\hskip \labelsep {\bfseries #1}\hskip \labelsep {\bfseries #2.}]}{\end{trivlist}}

\newenvironment{definition}[2][Definition]{\begin{trivlist}
    \item[\hskip \labelsep {\bfseries #1}\hskip \labelsep {\bfseries #2.}]}{\end{trivlist}}

\begin{document}

\title{Problem Set 4\\CS/MATH 113 Discrete Mathematics}
\author{Habib University | Spring 2023}
\date{Week 05}
\maketitle

\section{Problems}

\begin{problem}{1}
Explain what you must do to disprove the statement:
$x^3+5x + 3$ has a root between $x = 0$ and $x=1$
\end{problem}

\begin{problem}{2}
Prove that for any integer $n$ the number $n^2+5n+13$ is odd
\end{problem}

\begin{problem}{3}
State the statement of Contradiction and verify that it is a valid argument.\\
\textbf{Hint:} In contradiction we are saying that $A$ implies $B$ is the same as saying that $A$ and $\neg B$ happening together is false.
\end{problem}
\begin{problem}{4}
Show through contraposition the following proposition is true: $x \in \mathbb{Z}$. If $7x + 9$ is even, then $x$ is odd.
\end{problem}
\begin{problem}{5}
Prove that ``$(a+b)^2 = a^2 +b^2$'' is \textbf{not} an algebraic identity where $a,b \in \mathbb{R}$
\end{problem}
\begin{problem}{6}
Prove that for $m$ and $n$ integers, if 2 divides $m$ or 10 divides $n$, then 4 divides $m^{3}n^{2}$
\end{problem}
\begin{problem}{7}
Give a counterexample to the statement
\begin{center}
    ``If $n$ is an integer and $n^2$ is divisible by 4, then $n$ is divisible by 4''
\end{center}
\end{problem}
\begin{problem}{8}
Show through contraposition the following proposition is true : If $x^{2} - 6x + 5$ is even, then x is odd.
\end{problem}
\begin{problem}{9}
Show that any composite three-digit number must have a prime factor less than or equal to 31.
\end{problem}
\begin{problem}{10}
Show that if $a$ is a positive integer and $\sqrt[n]{a}$ is rational, then $\sqrt[n]{a}$ must be an integer.
\begin{problem}{11}
Prove the following claim: There exists irrational numbers $a$ and $b$ such that $a^b$ is rational.
\end{problem}
\begin{problem}{12}
Show that $\sqrt{2}$ is irrational. In other words, $\sqrt{2}$ cannot be written in the form $\frac{p}{q}$ where $p,q \in \mathbb{Z}$ and $q \neq 0$
\end{problem}

\begin{problem}{13}
Given that $p$ is a prime and $p|a^n$, prove that $p^n|a^n$. \end{problem}
\begin{problem}{14}
Show that there are infinitely many primes, in other words the set containing all prime numbers is infinite.
\end{problem}

\end{document}