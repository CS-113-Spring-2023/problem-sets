\documentclass[10pt]{article}

\usepackage{url}
 
\usepackage[margin=1in]{geometry} 
\usepackage{amsmath,amsthm,amssymb, graphicx, multicol, array}

 
\newcommand{\N}{\mathbb{N}}
\newcommand{\Z}{\mathbb{Z}}
 
\newenvironment{problem}[2][Problem]{\begin{trivlist}
\item[\hskip \labelsep {\bfseries #1}\hskip \labelsep {\bfseries #2.}]}{\end{trivlist}}

\newenvironment{definition}[2][Definition]{\begin{trivlist}
    \item[\hskip \labelsep {\bfseries #1}\hskip \labelsep {\bfseries #2.}]}{\end{trivlist}}

\begin{document}

\title{CS/MATH 113 -- Problem Set 2}
\author{Habib University -- Spring 2023}
\date{Week 02}
\maketitle

\section{Problems}

\begin{problem}{1}
Let $p$ and $q$ be the proposition
\begin{itemize}
    \item[$p:$] The election is decided.
    \item[$q:$] The votes have been counted.
\end{itemize}
Express each of these compound propositions as an English sentence.

\begin{itemize}
    \item[(a)] $\neg p$
    \item[(b)] $p \lor q$
    \item[(c)] $\neg p \land q $
    \item[(d)] $q \implies p $
    \item[(e)] $ \neg q \implies \neg p $
    \item[(f)] $\neg p \implies \neg q$
    \item[(g)] $p \leftrightarrow q$
    \item[(h)] $\neg q \lor (\neg p \land q) $
\end{itemize}
\end{problem}
\begin{problem}{2}
Let $p$ and $q$ be the proposition
\begin{itemize}
    \item[$p:$] It is below freezing.
    \item[$q:$] It is snowing.
\end{itemize}

Write these propositions using $p$ and $q$ and logical connectives (including negations)

\begin{itemize}
    \item [(a)] It is below freezing and snowing.
    \item [(b)] It is below freezing but not snowing.
    \item [(c)] It is not below freezing and it is not snowing.
    \item [(d)] It is either snowing or below freezing (or both).
    \item [(e)] If it is below freezing, it is also snowing.
    \item [(f)] Either it is below freezing or it is snowing, but it is
          not snowing if it is below freezing.
    \item [(g)] That it is below freezing is necessary and sufficient
          for it to be snowing.
\end{itemize}

\end{problem}

\begin{problem}{3}
Construct a truth table for each of these compound propositions.
\begin{itemize}
    \item[(a)] $ p \land \neg p$
    \item[(b)] $ p \lor \neg p$
    \item[(c)] $ (p \lor \neg q) \implies  q$
    \item[(d)] $ (p \lor q) \implies (p \land q) $
    \item[(e)] $ (p \implies q) \leftrightarrow (\neg q \implies \neg p) $
    \item[(f)] $ (p \implies q) \implies (p \implies q) $
\end{itemize}

\end{problem}

\begin{problem}{4}
In an Island there are two kinds of inhabitants, knights, who always tell the truth, and knaves, who always lie. You encounter two people
A and B. Determine, if possible, what A and B are if they address you in the ways described.
\begin{itemize}
    \item[(a)] A says ``At least one of us is a knave'' and B says nothing.
    \item[(b)] A says ``The two of us are both knights'' and B says ``A is a knave''. 
    \item[(c)] A says ``I am a knave or B is a knight'' and B says nothing.
    \item[(d)] Both A and B say ``I am a knight''
    \item[(e)] A says we ``We are both knaves '' and B says nothing. 
\end{itemize}
\end{problem}

\begin{problem}{5}
    Sheikh Chilly, famous for his bizarre sense of humor and love of logic puzzles, left the following clues
regarding the location of the hidden treasure. The treasure can only be in one place.
\begin{itemize}
    \item If the house is next to a lake, then the treasure is in the kitchen.
    \item If the house is not next to a lake or the treasure is buried under the flagpole, then the tree in the front yard is an elm and the tree in the back yard is not an oak.
    \item If the treasure is in the garage, then the tree in the back yard is not an oak.
    \item If the treasure is not buried under the flagpole, then the tree in the front yard is not an elm.
    \item The treasure is not in the kitchen.
\end{itemize}
Using rules of inference, determine where the treasure is hidden. Clearly state what your propositions
represent.
\end{problem}






\end{document}
